\documentclass[acmsmall, nonacm]{acmart}

% %%
%% \BibTeX command to typeset BibTeX logo in the docs
\AtBeginDocument{%
  \providecommand\BibTeX{{%
    Bib\TeX}}}

%% end of the preamble, start of the body of the document source.
\begin{document}

\title{Torchlight: Diffusion-based Traffic Generation on DARPA Searchlight}

\author{Ray Zhao}
\affiliation{%
 \institution{University of Southern California}
 \city{Los Angeles}
 \state{California}
 \country{}}
\email{rdzhao@usc.edu}

\author{Alefiya Hussain}
\affiliation{%
  \institution{USC/ISI}
  \city{Los Angeles}
  \country{}}
\email{hussain@isi.edu}

\renewcommand{\shortauthors}{Zhao, Hussain}

\begin{abstract}
  At present, there is a severe lack of both comprehensive and realistic labeled datasets
  for machine learning applications in the networking domain. Predominantly, prior generative
  work has focused on lower-dimensional representations that rely on aggregating flow 
  characteristics and lack the fine-grain of raw network traces. This results in suboptimal
  performance in machine learning contexts and limited applications outside of those 
  contexts. This has induced a push for new generative techniques to provide 
  synthetic data for usage both on its own and layered in with real data as augmentation.
  In this paper, we present Torchlight, a diffusion-based
  generation framework built atop and extending techniques first introduced in 
  \textit{NetDiffusion} \cite{Jiang2024} using DARPA Searchlight \cite{Ardi2022} data to generate synthetic network traces
  for video streaming applications. We demonstrate the efficacy of Torchlight in generating
  synthetic network traces that reasonably resemble real-world data and perform notably well
  in classification tasks. 
\end{abstract}

\maketitle

\section{Introduction}
ACM's consolidated article template, introduced in 2017, provides a
consistent \LaTeX\ style for use across ACM publications, and
incorporates accessibility and metadata-extraction functionality
necessary for future Digital Library endeavors. Numerous ACM and
SIG-specific \LaTeX\ templates have been examined, and their unique
features incorporated into this single new template.

If you are new to publishing with ACM, this document is a valuable
guide to the process of preparing your work for publication. If you
have published with ACM before, this document provides insight and
instruction into more recent changes to the article template.

The ``\verb|acmart|'' document class can be used to prepare articles
for any ACM publication --- conference or journal, and for any stage
of publication, from review to final ``camera-ready'' copy, to the
author's own version, with {\itshape very} few changes to the source.

\section{Motivation}
As noted in the introduction, the ``\verb|acmart|'' document class can
be used to prepare many different kinds of documentation --- a
double-anonymous initial submission of a full-length technical paper, a
two-page SIGGRAPH Emerging Technologies abstract, a ``camera-ready''
journal article, a SIGCHI Extended Abstract, and more --- all by
selecting the appropriate {\itshape template style} and {\itshape
  template parameters}.

This document will explain the major features of the document
class. For further information, the {\itshape \LaTeX\ User's Guide} is
available from
\url{https://www.acm.org/publications/proceedings-template}.

\subsection{Template Styles}

The primary parameter given to the ``\verb|acmart|'' document class is
the {\itshape template style} which corresponds to the kind of publication
or SIG publishing the work. This parameter is enclosed in square
brackets and is a part of the {\verb|documentclass|} command:
\begin{verbatim}
  \documentclass[STYLE]{acmart}
\end{verbatim}

Journals use one of three template styles. All but three ACM journals
use the {\verb|acmsmall|} template style:
\begin{itemize}
\item {\texttt{acmsmall}}: The default journal template style.
\item {\texttt{acmlarge}}: Used by JOCCH and TAP.
\item {\texttt{acmtog}}: Used by TOG.
\end{itemize}

The majority of conference proceedings documentation will use the {\verb|acmconf|} template style.
\begin{itemize}
\item {\texttt{sigconf}}: The default proceedings template style.
\item{\texttt{sigchi}}: Used for SIGCHI conference articles.
\item{\texttt{sigplan}}: Used for SIGPLAN conference articles.
\end{itemize}

\subsection{Template Parameters}

In addition to specifying the {\itshape template style} to be used in
formatting your work, there are a number of {\itshape template parameters}
which modify some part of the applied template style. A complete list
of these parameters can be found in the {\itshape \LaTeX\ User's Guide.}

Frequently-used parameters, or combinations of parameters, include:
\begin{itemize}
\item {\texttt{anonymous,review}}: Suitable for a ``double-anonymous''
  conference submission. Anonymizes the work and includes line
  numbers. Use with the \texttt{\acmSubmissionID} command to print the
  submission's unique ID on each page of the work.
\item{\texttt{authorversion}}: Produces a version of the work suitable
  for posting by the author.
\item{\texttt{screen}}: Produces colored hyperlinks.
\end{itemize}

This document uses the following string as the first command in the
source file:
\begin{verbatim}
\documentclass[acmsmall]{acmart}
\end{verbatim}

\section{Methods}

Modifying the template --- including but not limited to: adjusting
margins, typeface sizes, line spacing, paragraph and list definitions,
and the use of the \verb|\vspace| command to manually adjust the
vertical spacing between elements of your work --- is not allowed.

\section{Evaluation}

The ``\verb|acmart|'' document class includes the ``\verb|booktabs|''
package --- \url{https://ctan.org/pkg/booktabs} --- for preparing
high-quality tables.

Table captions are placed {\itshape above} the table.

Because tables cannot be split across pages, the best placement for
them is typically the top of the page nearest their initial cite.  To
ensure this proper ``floating'' placement of tables, use the
environment \textbf{table} to enclose the table's contents and the
table caption.  The contents of the table itself must go in the
\textbf{tabular} environment, to be aligned properly in rows and
columns, with the desired horizontal and vertical rules.  Again,
detailed instructions on \textbf{tabular} material are found in the
\textit{\LaTeX\ User's Guide}.

Immediately following this sentence is the point at which
Table~\ref{tab:freq} is included in the input file; compare the
placement of the table here with the table in the printed output of
this document.

\begin{table}
  \caption{Frequency of Special Characters}
  \label{tab:freq}
  \begin{tabular}{ccl}
    \toprule
    Non-English or Math&Frequency&Comments\\
    \midrule
    \O & 1 in 1,000& For Swedish names\\
    $\pi$ & 1 in 5& Common in math\\
    \$ & 4 in 5 & Used in business\\
    $\Psi^2_1$ & 1 in 40,000& Unexplained usage\\
  \bottomrule
\end{tabular}
\end{table}

To set a wider table, which takes up the whole width of the page's
live area, use the environment \textbf{table*} to enclose the table's
contents and the table caption.  As with a single-column table, this
wide table will ``float'' to a location deemed more
desirable. Immediately following this sentence is the point at which
Table~\ref{tab:commands} is included in the input file; again, it is
instructive to compare the placement of the table here with the table
in the printed output of this document.

\begin{table*}
  \caption{Some Typical Commands}
  \label{tab:commands}
  \begin{tabular}{ccl}
    \toprule
    Command &A Number & Comments\\
    \midrule
    \texttt{{\char'134}author} & 100& Author \\
    \texttt{{\char'134}table}& 300 & For tables\\
    \texttt{{\char'134}table*}& 400& For wider tables\\
    \bottomrule
  \end{tabular}
\end{table*}

Always use midrule to separate table header rows from data rows, and
use it only for this purpose. This enables assistive technologies to
recognise table headers and support their users in navigating tables
more easily.

\section{Figures}

The ``\verb|figure|'' environment should be used for figures. One or
more images can be placed within a figure. If your figure contains
third-party material, you must clearly identify it as such, as shown
in the example below.
\begin{figure}[h]
  \centering
  \includegraphics[width=\linewidth]{sample-franklin}
  \caption{1907 Franklin Model D roadster. Photograph by Harris \&
    Ewing, Inc. [Public domain], via Wikimedia
    Commons. (\url{https://goo.gl/VLCRBB}).}
  \Description{A woman and a girl in white dresses sit in an open car.}
\end{figure}


  % Some examples.  A paginated journal article \cite{Abril07}, an
  % enumerated journal article \cite{Cohen07}, a reference to an entire
  % issue \cite{JCohen96}, a monograph (whole book) \cite{Kosiur01}, a
  % monograph/whole book in a series (see 2a in spec. document)
  % \cite{Harel79}, a divisible-book such as an anthology or compilation
  % \cite{Editor00} followed by the same example, however we only output
  % the series if the volume number is given \cite{Editor00a} (so
  % Editor00a's series should NOT be present since it has no vol. no.),
  % a chapter in a divisible book \cite{Spector90}, a chapter in a
  % divisible book in a series \cite{Douglass98}, a multi-volume work as
  % book \cite{Knuth97}, a couple of articles in a proceedings (of a
  % conference, symposium, workshop for example) (paginated proceedings
  % article) \cite{Andler79, Hagerup1993}, a proceedings article with
  % all possible elements \cite{Smith10}, an example of an enumerated
  % proceedings article \cite{VanGundy07}, an informally published work
  % \cite{Harel78}, a couple of preprints \cite{Bornmann2019,
  %   AnzarootPBM14}, a doctoral dissertation \cite{Clarkson85}, a
  % master's thesis: \cite{anisi03}, an online document / world wide web
  % resource \cite{Thornburg01, Ablamowicz07, Poker06}, a video game
  % (Case 1) \cite{Obama08} and (Case 2) \cite{Novak03} and \cite{Lee05}
  % and (Case 3) a patent \cite{JoeScientist001}, work accepted for
  % publication \cite{rous08}, 'YYYYb'-test for prolific author
  % \cite{SaeediMEJ10} and \cite{SaeediJETC10}. Other cites might
  % contain 'duplicate' DOI and URLs (some SIAM articles)
  % \cite{Kirschmer:2010:AEI:1958016.1958018}. Boris / Barbara Beeton:
  % multi-volume works as books \cite{MR781536} and \cite{MR781537}. A
  % couple of citations with DOIs:
  % \cite{2004:ITE:1009386.1010128,Kirschmer:2010:AEI:1958016.1958018}. Online
  % citations: \cite{TUGInstmem, Thornburg01, CTANacmart}.
  % Artifacts: \cite{R} and \cite{UMassCitations}.

%%
%% The acknowledgments section is defined using the "acks" environment
%% (and NOT an unnumbered section). This ensures the proper
%% identification of the section in the article metadata, and the
%% consistent spelling of the heading.
\begin{acks}
To Alefiya, for the continual guidance and enthusiam all throughout
the project
\end{acks}

%%
%% The next two lines define the bibliography style to be used, and
%% the bibliography file.
\bibliographystyle{ACM-Reference-Format}
\bibliography{sample-base}

%%
%% If your work has an appendix, this is the place to put it.
\appendix
\end{document}
\endinput